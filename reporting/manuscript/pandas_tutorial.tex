\documentclass{article}

%%%%% Packages %%%%%%
\usepackage{authblk}
\usepackage{listings}
\usepackage{booktabs}
\usepackage{xcolor}

%%%%% Listings Style %%%%%
\definecolor{codegreen}{rgb}{0,0.6,0}
\definecolor{codegray}{rgb}{0.5,0.5,0.5}
\definecolor{codepurple}{rgb}{0.58,0,0.82}
\definecolor{backcolour}{rgb}{0.95,0.95,0.92}
\definecolor{deepblue}{rgb}{0,0,0.5}

\lstdefinestyle{mystyle}{
	language=Python,
	backgroundcolor=\color{backcolour},   
	commentstyle=\color{codegray},
	keywordstyle=\color{deepblue},
	numberstyle=\tiny\color{codegray},
	stringstyle=\color{codegreen},
	basicstyle=\ttfamily\footnotesize,
	breakatwhitespace=false,         
	breaklines=true,                 
	captionpos=b,                    
	keepspaces=true,                 
	numbers=none,                    
	numbersep=5pt,                  
	showspaces=false,                
	showstringspaces=false,
	showtabs=false,                  
	tabsize=2	
}
\lstset{style=mystyle}

%%%%% Preamble %%%%%
\title{Pandas Tutorial}
\author[ ]{Calderini, Matias}
\author[1]{X}
\author[2]{X}
\affil[1,2]{University of Ottawa}

%%%%% Document %%%%%
\begin{document}
	\maketitle
	\newpage
	
	\tableofcontents
	\listoftables
	\lstlistoflistings
	\newpage
	
	%%%%% Abstract %%%%%
	\section{Abstract}
	
	%%%%% Introduction %%%%%
	\section{Introduction}
	
	%%%%% Main Body %%%%%
	\section{Tutorial}
	Neeed to rename, but essentially the main body of the text
	
	%%%%% Setup %%%%% 
	\subsection{Setup}
	 Introduce how to download and get started?
	 
	%%%%% Libraries and Paths %%%%%
	\subsection{Libraries and Paths}
	
	%%%%% Reading and Saving Data
	\subsection{test}
	\begin{lstlisting}[caption=Sample]
		import numpy as np
		
		def incmatrix(genl1,genl2):
			m = len(genl1)
			n = len(genl2)
			M = None #to become the incidence matrix
			VT = np.zeros((n*m,1), int)  #dummy variable
			test = 'a string'
			
			#compute the bitwise xor matrix
			M1 = bitxormatrix(genl1)
			M2 = np.triu(bitxormatrix(genl2),1) 
			
			for i in range(m-1):
				for j in range(i+1, m):
					[r,c] = np.where(M2 == M1[i,j])
				for k in range(len(r)):
					VT[(i)*n + r[k]] = 1;
					VT[(i)*n + c[k]] = 1;
					VT[(j)*n + r[k]] = 1;
					VT[(j)*n + c[k]] = 1;
			
			if M is None:
				M = np.copy(VT)
			else:
				M = np.concatenate((M, VT), 1)
				
				VT = np.zeros((n*m,1), int)
		
		return M
	\end{lstlisting}
	
	%%%%% Dataframe %%%%%
	\section{The DataFrame - Understanding the data container}
	
	%%%%% Data %%%%%
	\section{The Data - Cleaning to avoid future headaches}
	\begin{tabular}{lrrrlr}
\toprule
{} &  year &    imdbid &  rating &                                   title &    id \\
\midrule
0 &  1888 &  392728.0 &       0 &                   Roundhay Garden Scene &  8040 \\
1 &  1892 &       3.0 &       0 &                          Pauvre Pierrot &  5433 \\
2 &  1895 &  132134.0 &       0 &  Execution of Mary, Queen of Scots, The &  6200 \\
3 &  1895 &      14.0 &       0 &           Tables Turned on the Gardener &  5444 \\
4 &  1896 &     131.0 &       0 &                       Une nuit terrible &  5406 \\
\bottomrule
\end{tabular}

	
	
	
\end{document}